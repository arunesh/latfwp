\newpage
\section{The Latitude DPOS Consensus Algorithm}
\label{app:dpos}

Latitude uses the Delegated Proof of Stake algorithm for consensus. This method allows honest participants and stake
holders to partake in block production. The same philosophy is also used for electing the Governance council. All
operations are transparent and any participant can become part of such operations through honest commitment to the
system. DPOS or variants thereof is being widely used in the blockchain space, current blockchains utilizing DPoS include:

\begin{itemize}
\item EOS, BitShares, Steem, Golos, Ark, Lisk, PeerPlays, Nano (formerly Raiblocks), and Tezos
\item Cosmos/Tendermint, Cardano, and a few others use consensus algorithms loosely based on DPoS
\end{itemize}

Latitude uses a combination of stake (amount of LAT token held as collateral) and trust (value in the trust ledger as
discussed in Section \ref{sec:trust}. The block production algorithm works as follows:

\begin{itemize}
\item Participants who own a certain amount of trust and stake are allowed to act as delegates. Trust and stake are
    combined into a single metric called the l-factor as given in Equation \ref{eq_l_factor}.
\item Delegates can choose to nominate block producers. Their vote is weighted by their l-factor value. The block
    candidates who receive the most votes are elected to be block producers. Users can also delegate their voting power
        to others on their behalf. Thus DPOS creates a representative democracy with stake and trust based suffrage.
\item The number of block producers is lower bounded $N_{min}$ and upper bounded $N_{max}$, but is not a strict value.

\item Block producers can be voted out any time if bad behavior is determined. The threat of loss of stake and/or
    reputation is a deterren to bad behavior. Also slashing conditions can be easily implemented if so desired.
\item Block consensus: Once a block producer set is determined, a consensus algorithm such as PBFT is used to
    achieve $2/3 +1$ majority to decide on a block. Producers always converge to the longest chain as as long as a
        majority of the block producers are honest, the system will converge.
\item Block production time: This can be set to a desirable value, such as 0.5 or 1 second to make it periodic.
\end{itemize}

\newpage
\section{Scaling the Latitude Architecture}
\label{app:latchain}

As the Latitude platform grows both in terms of aggregate data/users or partners, it might become necessary to split
each application vertical into its own side-chain. A Latitude base-chain will work with various side-chains to enforce
basic platform rules such as governance, token pricing, platfrom-wide smart contracts, etc.  
