\newpage
\section{The Latitude DPOS Consensus Algorithm}
\label{app:dpos}

Latitude uses the Delegated Proof of Stake algorithm for consensus. This method allows honest participants and stake
holders to partake in block production. The same philosophy is also used for electing the Governance council. All
operations are transparent and any participant can become part of such operations through honest commitment to the
system. DPOS or variants thereof is being widely used in the blockchain space, current blockchains utilizing DPoS include:

\begin{itemize}
\item EOS, BitShares, Steem, Golos, Ark, Lisk, PeerPlays, Nano (formerly Raiblocks), and Tezos
\item Cosmos/Tendermint, Cardano, and a few others use consensus algorithms loosely based on DPoS
\end{itemize}

Latitude uses a combination of stake (amount of LAT token held as collateral) and trust (value in the trust ledger as
discussed in Section \ref{}

\newpage
\section{Scaling the Latitude Architecture}
\label{app:latchain}

As the Latitude platform grows both in terms of aggregate data/users or partners, it might become necessary to split
each application vertical into its own side-chain. A Latitude base-chain will work with various side-chains to enforce
basic platform rules such as governance, token pricing, platfrom-wide smart contracts, etc.  
